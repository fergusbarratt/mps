\documentclass{article}

\usepackage{physics}
\usepackage{amsmath}
\usepackage[margin=1in]{geometry}

\title{Transfer Report}
\author{Fergus Barratt}

\begin{document}
\maketitle
\section{Introduction}
The effective use of quantum technology requires the systems we build to be isolated from the environment.
In general this can only be achieved approximately.
In this project, we investigate the role of the world at large in three different ways
\begin{enumerate}
    \item In gate-based quantum computation, the threshold theorem says that any local environmental influence can be overcome by error correction.
          No such theorem exists for adiabatic quantum computation. 
          We introduce tools from classical statistical mechanics to characterise the failure of adiabatic quantum computation.
    \item We introduce a useful connection between different ways of treating the environment.
    \item Classical systems exhibit dynamical chaos - an exponential sensitivity to initial conditions. 
          In quantum systems classical chaos is impossible.
          What role does the standard tool for characterising dynamical chaos - the lyapunov spectrum - play in the quantum case?
          We build on some earlier work in this area, and compare to standard tools for evaluating quantum chaos.
\end{enumerate}
%
\section{Failure of Adiabatic Quantum Computation}\label{sec:failure}
%
\subsection{Introduction}
Adiabatic Quantum Computation (AQC) is an alternative (equivalent~\cite{Aharonov2004}) paradigm of quantum computation.
We prepare the ground state of some easily accessed Hamiltonian.
If we tune the parameters in the hamiltonian sufficiently slowly, then by the adiabatic theorem~\cite{born1928adiabatic} the system will remain in the ground state of the time varying hamiltonian.
By transforming the state in this way we can perform a quantum computation.

Whilst the effect of the environment on gate-based quantum computation is well understood, the same is not true about AQC.
In particular, for gate-based QC we have threshold theorems that suggest that if we have access to sufficiently many qubits we can overcome the effects of local noise. 
In the following, we introduce methods popularised in the study of classical glasses to characterise failure in AQC.
\subsection{Dynamical Phase Transitions}
Phase transitions are conventionally understood by a free energy, which becomes discontinuous as a function of some system parameter. 
One standard example is found in the classical Ising model in two dimensions, in which we can distinguish a paramagnetic from a ferromagnetic phase by a discontinuity in the free energy as a function of temperature.
Each phase is characterised by the value of some order parameter, in this case the magnetisation.
A more rigorous understanding of this formalism can be achieved by using large deviation theory. 
The partition function 
\begin{equation}
    \mathcal{Z}_N(\beta) = \ev{e^{-\beta H}}
\end{equation}
is the moment generating function (MGF) of the energy in the probability distribution over microstates.
The free energy 
\begin{equation}
    f(\beta) = \lim_{N\rightarrow\infty} \frac{1}{N} \mathcal{Z}_N(\beta)
\end{equation}
is the \emph{scaled cumulant generating function} (sCGF) associated with this probability distribution.

AQC is a dynamical process, and the failure of AQC is a dynamical transition - computational trajectories which support sufficient entanglement succeed, and those that don't, fail. 
If this failure can be characterised by an order parameter, it must be dynamical, that is must depend on a whole trajectory. 
In studies of the glass transition~\cite{Garrahan2007}, a formalism was introduced for characterising just this kind of transition.
We define the \emph{dynamical partition function}
\begin{equation}
    \mathcal{Z}_t(s) = \ev{e^{-sQ}}\label{eq:mgf}
\end{equation}
where $Q = \int^t \dd{t'} I(t')$ is some time-extensive (in this case time-integrated) observable, and the average is taken over an ensemble of trajectories.
$s$ is a generating parameter, of which more later.
We can thus define:
\begin{equation}
    \varphi(s) = \lim_{t\rightarrow\infty} \frac{1}{t} \ln\mathcal{Z}_t(s), \label{eq:cgf}
\end{equation}
the \emph{dynamical free energy}.
Dynamical phase transitions are discontinuities in the dynamical free energy, and seperate phases characterised by a dynamical order parameter.
%
\subsubsection{Bias}
%
In the static case, the free energy depends on the inverse temperature $\beta$, which has a sensible physical meaning. 
In the dynamical case the interpretation of the parameter $s$ is not clear. 
We can understand $\beta$ as a bias on the microcanonical ensemble, making certain states more or less probable, depending on their energy.
In the same way $s$ makes certain trajectories more or less probable, depending on the value of $Q$.

In AQC, it is the environment that biases trajectories - for stronger dissipation, trajectories with less entanglement are more likely. 
We will show that these biasing effects are one and the same - that is, environmental bias can be understood as a bias on an ensemble of quantum trajectories. 
%
\subsubsection{Quantum Mechanics}
%
There are subtleties associated with understanding eqs.~\ref{eq:mgf} and~\ref{eq:cgf} for quantum systems. 
In particular, given that measurements in quantum mechanics are commonly understood as single shot operations, how can the exponent of eq.~\ref{eq:mgf} be understood?
The answer is provided by the full counting statistics~\cite{Nazarov2001}.
In Ref~\cite{Hickey2013} the $s$-biased states are introduced
\begin{equation}
    \ket{s} = \flatfrac{e^{i(H-is\hat{I})t} \ket{\Omega}}{\norm{e^{i(H-is\hat{I})t} \ket{\Omega}}}
\end{equation}
in terms of which 
\begin{equation}
    \varphi(s) = -\ev{k}{s}/s
\end{equation}
\subsection{The Failure of Adiabatic Quantum Computation as a Dynamical Phase Transition}
We apply the above ideas in a simple model of AQC, and show the equivalence of trajectory bias and environmental influence.
%
\subsubsection{Two Spin model}
%
\begin{itemize}
    \item Introduce the two spin model.
    \item Justify $h=0$.
    \item Quantum trajectories determine AQC success.
    \item Introduce Dynamical phase transition.
\end{itemize}
%
\subsubsection{Parametrisation of two spin Hilbert Space}
%
\begin{itemize}
    \item Derive parametrisation from Schmidt decomposition
    \item Derive EOM from heisenberg equations.
\end{itemize}
%
\subsubsection{Zero magnetisation}
%
\begin{itemize}
    \item Initial state restricts to zero magnetisation subspace
    \item EOM for restricted subspace
\end{itemize}




\section{Keldysh Field Theory \& the Lindblad Equation}\label{sec:open}
%
\subsection{Introduction}
The Lindblad equation has become the standard tool for treating Markovian open quantum systems. 
Keldysh Field Theory allows us to write field theoretic descriptions of open quantum systems. 
In general the two represent distinct ways of describing open quantum systems - nevertheless, for consistency they must each be derivable from the other. 
In the following we derive the Lindblad Equation from a Keldysh Field Theory.
%
\section{Lyapunov Spectra in Matrix Product State Simulations}\label{sec:chaos}
%
\subsection{Introduction}
According to the classical definition, chaos is precluded in quantum systems by the unitarity of the Schr\"odinger equation.
Quantum evolution does show signatures of chaos however, and the role of classical chaos in i.e.\ thermalization is performed in other ways.
In this section we build on the work of Ref.~\cite{Andrew}, in which a connection is made between classical chaotic dynamics and thermalisation via Lyapunov spectra extracted in several semiclassical limits.
\subsubsection{Lyapunov Exponents and the Lyapunov Spectrum}
The Lyapunov exponents characterise the rate of growth of infinitesimal perturbations to an initial condition.
\subsubsection{Matrix Product States and the Time Dependent Variational Principle}
Matrix Product States (hereinafter MPS) provide a convenient parametrisation of wavefunctions in 1 dimension.
\subsection{Results}

\bibliographystyle{unsrt}
\bibliography{bibliography}
\end{document}
